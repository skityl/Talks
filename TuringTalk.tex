\documentclass[pdf]{beamer}

\usepackage{mathtools}
\usepackage{tikz}
\usepackage{animate}
\usepackage{wrapfig}
\usepackage{listings}
\usepackage{color}
\usetheme{Dresden}
\usecolortheme{seahorse}


\newtheorem{principle}{Principle}
\newtheorem{proposition}{Proposition}
\definecolor{dkgreen}{rgb}{0,0.6,0}
\definecolor{gray}{rgb}{0.5,0.5,0.5}
\definecolor{mauve}{rgb}{0.58,0,0.82}

\lstset{frame=tb,
  language=Java,
  aboveskip=3mm,
  belowskip=3mm,
  showstringspaces=false,
  columns=flexible,
  basicstyle={\small\ttfamily},
  numbers=none,
  numberstyle=\tiny\color{gray},
  keywordstyle=\color{blue},
  commentstyle=\color{dkgreen},
  stringstyle=\color{mauve},
  breaklines=true,
  breakatwhitespace=true
  tabsize=3
}

\newcommand{\laplace}[1]{ \mathcal{L} \left\{ #1 \right\} }
\newcommand{\fourier}[1]{ \mathcal{F} \left\{ #1 \right\} }
\newcommand{\mellin}[1]{ \mathcal{M} \left\{ #1 \right\} }
\newcommand{\rld}[4]{ \left( \prescript{}{#1}{\mathcal{D}_{#2}^{#3}} #4 \right) }
\newcommand{\rli}[3]{ \left( I_{#1}^{#2} #3 \right) }
\newcommand{\der}[3]{ \frac{d^{#3}#1}{d#2^{#3}} }
\newcommand{\capder}[4]{ \left( \prescript{C}{#1}{\mathcal{D}_{#2}^{#3}} #4 \right) }
\newcommand{\fracdelta}[4]{ \left( \prescript{}{#1}{\Delta^{#2}_{#3} } #4 \right) }

\newcommand{\lra}{\longrightarrow}
\newcommand{\ra}{\rightarrow}
\newcommand{\lla}{\longleftarrow}
\newcommand{\la}{\leftarrow}

%Analysis
\newcommand{\Rl}{\mathbb{R}}
\newcommand{\Cplx}{\mathbb{C}}
\newcommand{\Itgr}{\mathbb{Z}}
\newcommand{\Ntrl}{\mathbb{N}}
\newcommand{\Ind}{\mathbbm{1}}
\newcommand{\Hlbt}{\mathcal{H}}
\newcommand{\im}{\operatorname{im}}

\mode<presentation>{}
\title{Formation of Turing Patterns}
\subtitle{A Historical Introduction}
\author[Anna McGann]{Anna McGann\\{\small Supervised by: Dr Chris Angstmann and Prof Bruce Henry}}
\institute{
	School of Mathematics and Statistics \\
	University of New South Wales
}

\begin{document}
\begin{frame}
	\titlepage
\end{frame}

\begin{frame} {Original Motivations}

\begin{figure}
\includegraphics[scale = 0.3]{zebra3}
\caption{A Turing pattern wave front generated over a zebra}
\end{figure}
\end{frame}


%\begin{frame}{Outline}
%\begin{columns}[T]
 %   \begin{column}{.5\textwidth}
  %   \begin{block}{}
%\begin{itemize}
%        \item Turing Instability 
 %       \item Original Motivations
  %      \item Turing Patterns on Networks and Future Plans
   % \end{itemize}
%    \end{block}
 %   \end{column}
  %  \begin{column}{.5\textwidth}
   % \begin{block}{}
%    \begin{figure}
 %   \includegraphics[scale=0.35]{Alan}
  %  \end{figure}
   % \end{block}
    %\end{column}
%  \end{columns}
%\end{frame}

\begin{frame}{Turing Instability}
	A Turing Instability is a diffusion driven instability, in which the spatially homogeneous steady state solutions lose their stability in the presence of diffusion.
	
	\begin{columns}[T]
    \begin{column}{.5\textwidth}
     \begin{block}{General Equations}
\begin{align*}
        \frac{\partial u}{\partial t} = f_1(u,v) +  D_1\frac{\partial^2 u}{\partial x^2} \\
        \frac{\partial v}{\partial t} = f_2(u,v) + D_2\frac{\partial^2 v}{\partial x^2} \\ 
    \end{align*}
    \end{block}
    \end{column}
    \begin{column}{.5\textwidth}
    \begin{block}{Concentrations}
    \begin{figure}[l]
    \includegraphics[scale=0.3]{concentrations}
    \caption{Taken from Aubert, 2014}
    \end{figure}
    \end{block}
    \end{column}
  \end{columns}
\end{frame}

\begin{frame}{Stability Analysis}
	Now looking at
\begin{align}	
	\mathbf{u} =
\left(		
	\begin{array}{c}
	u \\
	v
	\end{array}
	\right)
	\end{align}
	Now linearising around the steady states $ u_* $	
	\begin{equation}
		\mathbf{u} = \mathbf{u}_*+ \triangle \mathbf{u}	
	\end{equation}
	Here the steady state solutions are of the form, with no diffusion:
	\begin{equation}
	\triangle \mathbf{u} = \sum_j a_j e^{\lambda_j t}\underline{v}^{(j)}
	\end{equation}
	And with diffusion:
	\begin{equation}
	\triangle \mathbf{u} = \sum_j\sum_k a_j e^{\lambda_k t}\cos(\frac{k\pi x}{L})
	\end{equation}
	%\begin{align*}
	%	\left(		
		%\begin{array}{cc}        
        %\frac{\partial f_1}{\partial n_1} & \frac{\partial f_1}{\partial n_2} \\
        %\frac{\partial f_2}{\partial n_1} & \frac{\partial f_2}{\partial n_2}
      %  \end{array} 
       % \right)
%		\ \ \ M =
%		\left(
%		\begin{array}{cc}        
 %       (\frac{\partial f_1}{\partial n_1} - D_1\frac{k^2 \pi^2}{L^2}) & \frac{\partial f_1}{\partial n_2} \\
  %      \frac{\partial f_2}{\partial n_1} & (\frac{\partial f_2}{\partial n_2} - D_2\frac{k^2 \pi^2}{L^2})
   %     \end{array} 
    %    \right)
%    \end{align*}
 %   \begin{align*}
  %      \operatorname{tr}\ A< 0 \ \ \ \ \ \ \ \ \ \ \ \ \  \ \ \operatorname{tr}\ M > 0 \\
   % 	\det \ A> 0 \ \ \ \ \ \ \ \ \ \ \ \ \  \ \operatorname{det} \ M < 0
   % \end{align*}
\end{frame}



\begin{frame}{Activator-Inhibitor Systems}
	Gierer-Meinhardt Equations (1972)
		\begin{align}
    		\frac{\partial u}{dt} = \rho \frac{u^2}{v} - \mu_u u + D_u \frac{\partial^2 u}{\partial x^2} + \rho_u \\
    		\frac{\partial v}{dt} = \rho u^2 - \mu_v v + D_v \frac{\partial^2 v}{\partial x^2} + \rho_v
    	\end{align}
    Here $ u $ is an activator, while $ v $ is an inhibitor
\end{frame}

\begin{frame}{Activator-Inhibitor Systems}
   \begin{center}
\animategraphics[scale=0.32]{1}{GM_c}{1}{81}
   \end{center}
   Contour plot shows the variation in the concentration of the activator as the Gierer-Meinhardt equations are perturbed. Following the parameters:
   \begin{align*}
\rho = 0.1, \mu_u=0.18, \rho_u=0.2, D_u=0.055  \\   
   \end{align*}
\end{frame}

\begin{frame}{Activator-Inhibitor Example}
\begin{columns}[T]
    \begin{column}{.5\textwidth}
     \begin{block}{Zebra Fish}
\begin{itemize}
\item Zebra fish made up of yellow and black
\item Black is the activator, yellow inhibitor
\item Pigment was surgically removed and regrowth documented
\item Regrowth matched Turing model predictions
\end{itemize}
    \end{block}
    \end{column}
    \begin{column}{.5\textwidth}
    \begin{block}{Pigment regrowth}
    \begin{figure}[l]
    \includegraphics[scale=0.24]{zebra}
    \caption{Taken from Kondo, 2010}
    \end{figure}
    \end{block}
    \end{column}
  \end{columns}
\end{frame}

\begin{frame}{Motivations for Networks}
\begin{itemize}
\item Diffusion over the whole space only allows for geographical diffusion
\item Looking at Turing patterns develop over networks allows us to see impact of topology of the network
\item This is important in the case of non-local diffusion and other transport networks
\end{itemize}
\end{frame}

\begin{frame}{Extension to Networks}
When we apply these equations to networks, the diffusive term is not well defined and hence is evaluated as a Continuous Time Markov Chain.
\begin{align}
        \frac{\partial u}{\partial t} = f_1(u,v) +  D_1\nabla^2\cdot u
\end{align}
\begin{align}
        \frac{\partial u}{\partial t} = f_1(u,v) + \sum_{j=1}^N \lambda_u(i|j,t)\alpha_u(j)u(j,t) -\alpha_u(i)u(i,t)
\end{align}
\end{frame}

\begin{frame}{Extension to Networks}
\begin{figure}
\includegraphics[scale=0.5]{network}
\caption{Taken from Angstmann et al., 2013, Concentrations of 2 reactants over a network's nodes}
\end{figure}
\end{frame}

\begin{frame}{Conclusion}
	\begin{itemize}
		\item Reaction-Diffusion Equations
		\item Seen very often in nature
		\item Looking at diseases causing these instabilities
	\end{itemize}
\end{frame}

\begin{frame}{Key References}
\begin{itemize}
\item [1]Turing, Alan Mathison. "The chemical basis of morphogenesis." Philosophical Transactions of the Royal Society of London. Series B, Biological Sciences 237.641 (1952): 37-72.
\item [2]Gierer, Alfred, and Hans Meinhardt. "A theory of biological pattern formation." Kybernetik 12.1 (1972): 30-39.
\item [3]Nakamasu, Akiko, et al. "Interactions between zebrafish pigment cells responsible for the generation of Turing patterns." Proceedings of the National Academy of Sciences 106.21 (2009): 8429-8434.
\end{itemize}
\end{frame}
\begin{frame}{Key References}
\begin{itemize}
\item [4] Angstmann, C. N., et al. "Continuous-time random walks on networks with vertex-and time-dependent forcing." Physical Review E 88.2 (2013): 022811.
\item [5] Kondo, Shigeru, and Takashi Miura. "Reaction-diffusion model as a framework for understanding biological pattern formation." Science 329.5999 (2010): 1616-1620.
\item [6] Turk, Greg. Zebra3, 1997, http://www.cc.gatech.edu
\item [7] Aubert, Nathanaël, et al. "Computer-assisted design for scaling up systems based on DNA reaction networks." Journal of The Royal Society Interface 11.93 (2014): 20131167.
\end{itemize}
\end{frame}


\end{document}
